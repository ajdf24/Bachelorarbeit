\section{Zusammenfassung}
Die vorliegende Arbeit, besch\"aftigte sich mit der Implementierung eines system\"ubergreifenden Dokumentenmanagementsystems.
Hierf\"ur im Verlauf der Arbeit als erstes ein Lastenheft erstellt, welches alle Anforderungen f\"ur einen ersten Prototyp festh\"alt.
Es wurde auf den zuk\"unftigen Produkteinsatz, sowie auf Produktfunktionen, Produktdaten und Produktleistungen genauer eingegangen.

F\"ur die Entwicklung eines neuen Systems, welches Altsysteme abl\"o\ss{}en soll, mussten die bisher bestehenden Systeme genauer analysiert werden. Daf\"ur wurden die Internetportale \ac{FADO}, \ac{DRS} und das Naturschutz-Bildarchiv der LUBW genauer analysiert (siehe Kapitel \ref{Stand der Technik}). Zus\"atzlich wurde auch das Literaturarchiv der \ac{ICT-ENSURE} in die Analyse einbezogen.

Im Kapitel \ref{Stand der Technik} wurden nach einer oberfl\"achlichen Analyse die Metadaten aller Systeme gesammelt und aufgelistet. Dies war notwendig, da im sp\"ateren Verlauf der Arbeit ein \"ubergreifenden Metadatenmodell aller Systeme erstellt wurde.

Da die Europ\"aische Union erlassen hat, dass jede vom Land neu ver\"offentlichte Datei einen Geodatenbezug haben muss, wurde auch in dieser Richtung analysiert. So entstand im Kapitel \ref{Analyse Datenbestaende} eine Unterteilung zwischen fachlichen Metadatenstandards und technischen Metadatenstandards. Hier wurden mehrere Metadatenstandards genauer analysiert, um ach sie in das \"ubergreifende Metadatenmodell einbeziehen zu k\"onnen.

Nachdem nun alle Vorrausetzungen gegebenen waren, konnte eine \"ubergreifendes Metadatenmodell zu erstellen, welches hochmodular gehalten ist. Durch diese Modularit\"at, ist es relativ einfach das Modell zu erweitern und neu Dokumententypen k\"onnen sch\"on vorhandene Untergruppen nutzen. Im Kapitel \ref{Erstellung eines Datenkonzepts} wird das im Rahmen dieser Arbeit erstellte Modell genauer betrachtet und erkl\"art.

Als die theoretischen Grundlagen f\"ur die Erstellung eines Prototyps gegeben waren, konnten verschiedene ECM-Systeme verglichen und auf ihre Projekttauglichkeit gepr\"uft werden. In der Arbeit wurden die vier ECM-Systeme "`agorum core"', "`Alfresco"', "`Open-Xchange"' und "`Liferay"' als ECM-Tool n\"aher betrachtet. Auch wenn Liferay im eigentlichen Sinn ein Portalserver ist, so besitzt auch er eine Dokumentverwaltung, welche als ECM-System gesehen werden kann. 

Nun da die vier Systeme beschrieben und gepr\"uft wurden sind, wurde eine Auswertung nach verschiedenen Kriterien aus dem Lastenheft erstellt. Hierbei ging Alfresco eindeutig als Sieger hervor, weshalb in der weiteren Bearbeitung dieses ECM-System f\"ur die Realisierung eines Backends gew\"ahlt wurde (siehe Kapitel \ref{Technologievergleich}).

Nachdem nun ein Backendserver ausgew\"ahlt ware, musse das zuvor erstellte allgemeine Metadatenmodell an Alfresco angepasst werden. Dies war notwendig, da Alfresco keine Unterklassen von Metadaten als Attribute zul\"asst. Des weiteren hat Alfesco gewissen Metadatenstandards wie das \ac{Exif}-Format oder Dublin Core schon standardm\"a\ss{}ig implementiert. Daher entfiel eine Implementierung dieser Standards innerhalb des Metadatenmodells.

Als das Metadatenmodell angepasst war, wurde es in Alfresco implementiert. Hierf\"ur mussten, wie in der Arbeit im Kapitel \ref{Implementierung Backend} erl\"autert einige XML-Dateien angelegt und bearbeitet werden. Zum einen werden Dateien ben\"otigt, welche das Datenmodell grundlegend beschreiben, zum anderen werden auch Dateien ben\"otigt, die eine korrekte Darstellung der Metadaten innerhalb von Alfresco erm\"oglichen. 
Das Metadatenmodell wird dann beim Start des Tomcat-Servers als Java-Beans geladen und kann verwendet werden.


