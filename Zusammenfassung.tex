\section{Zusammenfassung}
Die vorliegende Arbeit besch\"aftigte sich mit der Implementierung eines system\"ubergreifenden Dokumentenmanagementsystems.
Hierf\"ur wurde im Verlauf der Arbeit als erstes ein Lastenheft erstellt, welches alle Anforderungen f\"ur einen ersten Prototyp festh\"alt.
Es wurde auf den zuk\"unftigen Produkteinsatz, sowie auf Produktfunktionen, Produktdaten und Produktleistungen genauer eingegangen.

F\"ur die Entwicklung eines neuen Systems, welches Altsysteme abl\"osen soll, mussten die bisher bestehenden Systeme genauer analysiert werden. Daf\"ur wurden die Internetportale \ac{FADO}, \ac{DRS} und das Naturschutz-Bildarchiv der LUBW genauer analysiert (siehe Kapitel \ref{Stand der Technik}). Zus\"atzlich wurde auch das Literaturarchiv der \ac{ICT-ENSURE} in die Analyse einbezogen.

Im Kapitel \ref{Stand der Technik} wurden nach einer oberfl\"achlichen Analyse die Metadaten aller Systeme gesammelt und aufgelistet. Dies war notwendig, da im sp\"ateren Verlauf der Arbeit ein \"ubergreifenden Metadatenmodell aller Systeme erstellt wurde.

Da die Europ\"aische Union erlassen hat, dass jede vom Land neu ver\"offentlichte Datei einen Geodatenbezug haben muss, wurde auch in dieser Richtung analysiert. So entstand im Kapitel \ref{Analyse Datenbestaende} eine Unterteilung zwischen fachlichen Metadatenstandards und technischen Metadatenstandards. Hier wurden mehrere Metadatenstandards genauer analysiert, um auch sie in das \"ubergreifende Metadatenmodell einbeziehen zu k\"onnen.

Nachdem nun alle Vorrausetzungen gegebenn waren, konnte ein \"ubergreifendes Metadatenmodell erstellt werden, welches hochmodular gehalten ist. Durch diese Modularit\"at ist es relativ einfach, das Modell zu erweitern und neue Dokumententypen k\"onnen schon vorhandene Untergruppen nutzen. Im Kapitel \ref{Erstellung eines Datenkonzepts} wird das im Rahmen dieser Arbeit erstellte Modell genauer betrachtet und erkl\"art.

Als die theoretischen Grundlagen f\"ur die Erstellung eines Prototyps gegeben waren, konnten verschiedene ECM-Systeme verglichen und auf ihre Projekttauglichkeit gepr\"uft werden. In der Arbeit wurden die vier ECM-Systeme "`agorum core"', "`Alfresco"', "`Open-Xchange"' und "`Liferay"' als ECM-Tool n\"aher betrachtet. Auch wenn Liferay im eigentlichen Sinn ein Portalserver ist, so besitzt auch er eine Dokumentverwaltung, welche als ECM-System gesehen werden kann. 

Nun, da die vier Systeme beschrieben und gepr\"uft worden sind, wurde eine Auswertung nach verschiedenen Kriterien aus dem Lastenheft erstellt. Hierbei ging Alfresco eindeutig als Sieger hervor, weshalb in der weiteren Bearbeitung dieses ECM-System f\"ur die Realisierung eines Backends gew\"ahlt wurde (siehe Kapitel \ref{Technologievergleich}).

Nachdem nun ein Backendserver ausgew\"ahlt war, musste das zuvor erstellte allgemeine Metadatenmodell an Alfresco angepasst werden. Dies war notwendig, da Alfresco keine Unterklassen von Metadaten als Attribute zul\"asst. Des Weiteren hat Alfesco gewisse Metadatenstandards wie das \ac{Exif}-Format oder Dublin Core schon standardm\"a\ss{}ig implementiert. Daher entfiel eine Implementierung dieser Standards innerhalb des Metadatenmodells.

Als das Metadatenmodell angepasst war, wurde es in Alfresco implementiert. Hierf\"ur mussten, wie in der Arbeit im Kapitel \ref{Implementierung Backend} erl\"autert, einige XML-Dateien angelegt und bearbeitet werden. Zum einen werden Dateien ben\"otigt, welche das Datenmodell grundlegend beschreiben, zum anderen werden auch Dateien ben\"otigt, die eine korrekte Darstellung der Metadaten innerhalb von Alfresco erm\"oglichen. 
Das Metadatenmodell wird dann beim Start des Tomcat-Servers als Java-Beans geladen und kann verwendet werden.

Eine \"Ubesetzung des Metadatenmodells wurde aus Zeitgr\"unden nur prototypisch umgesetzt, um die Funktionalit\"at zu demonstrieren. 

Die Integration der Alfresco-Daten in den Portalserver Liferay konnte nicht umgesetzt werden, da es keine fertige Implementierung f\"ur diesen Use-Case gibt. Liferay ist weder standardm\"a\ss{}ig noch \"uber ein Plugin in der Lage, die Metadaten von Alfresco abzufragen und zu speichern.

Im Kapitel \ref{Implementierung Frontend} wurden M\"oglichkeiten vorgestellt, wie eine nachtr\"agliche Bereitstellung der gew\"unschten Funktionalit\"at m\"oglich ist. Die beste M\"oglichkeit ist die Implementierung eines Hooks, welcher es erlaubt, die Metadaten aus Alfresco auszulesen und sie in der Liferay-Datenbank zu speichern. Dies wurde prototypisch implementiert, um die Machbarkeit zu demonstrieren. Eine Anzeige- und Suchfunkion wurde nicht realisiert, ist aber nachdem die Daten in der Liferay-Datenbank nun vorhanden sind, nicht mehr abwegig und sollte relativ leicht zu Implementieren sein.

\subsection{Umsetzung des Lastenhefts}
Ziel der Arbeit war es, wie im Lastenheft beschrieben, ein Backend f\"ur die Verwaltung von Dokumenten der \ac{LUBW} zu erstellen. Dies konnte auch erfolgreich umgesetzt werden. Das Frontend, welches auf Liferay basieren sollte, war kritischer, da Liferay nicht in der Lage ist, Metadaten von Alfresco zu verarbeiten. Es wurde zwar eine prototypische Erweiterung f\"ur Liferay erstellt, den vollen Funktionsumfang bietet sie jedoch nicht.

Es konnten somit bis auf die Produktfunktionen "`LF40"' und "`LF60"' alle Anforderungen erf\"ullt werden. Die Metadatensuche in Liferay konnte aus Zeitgr\"unden nicht umgesetzt werden. Da ein Anzeigen der Metadaten auch aus Zeitgr\"unden nicht m\"oglich war, konnte der Prototyp nicht mit den alten \ac{LUBW}-Systemen verglichen werden.

Die im Lastenheft festgelegten Produktdaten (siehe Abschnitt \ref{Produktdaten}) beziehen sich alle auf das Backend und konnten komplett umgesetzt werden. Im Verlauf der Arbeit wurde ein modulares Metadaten-Modell entwickelt, welches den Anforderungen des Lastenhefts fast gen\"ugt. Alfresco erlaubt es nicht, Metadaten zu erstellen, welchen keine physische Datei zugrunde liegt. Somit muss ein Autor, wie in "`LD30"' beschrieben, als Attribut gespeichert werden und kann nicht als Entit\"at existieren.

Existierende Standards wurden im Verlauf der Arbeit betrachtet und auch implementiert. Hier war vorteilhaft, dass Alfresco schon manchen Standard implementiert hat, so zum Beispiel den "`Dublin Core"'- oder den "`\ac{Exif}"'-Standard.

Die beschriebenen Produktleistungen konnten zwar f\"ur das Backend umgesetzt werden, jedoch aus den schon beschriebenen Gr\"unden nicht f\"ur das Frontend. Somit konnte der Punkt "`LL30"' nur zum Teil umgesetzt werden. Allen anderen Punkte konnte jedoch Gen\"uge getan werden.


