\section{Anhang} \label{Anhang}
\subsection{Metadaten der LUBW Fachsysteme} \label{Metadaten der LUBW Fachsysteme}
In den folgenden Abschnitten sind die Metadaten und Relationen der Untersuchten Dokumente in \ac{FADO} abgebildet.
\subsubsection{Metadaten des Fachsystems FADO / Urteile}
\begin{table}[ht]
\begin{center}
\begin{tabular}{|l|l|}
\hline
\textbf{Metadaten:} & \textbf{Werte:} \\ \hline
Fachsystem & Vorgabe \\ \hline
ID & Generiert \\ \hline
Titel & Freitext \\ \hline
Tenor & Freitext \\ \hline
Kommentar & Freitext \\ \hline
Orientierungssatz & Freitext \\ \hline
Norm & Freitext \\ \hline
Leitsatz & Freitext \\ \hline
Gericht & Freitext \\ \hline
Entscheidungsform & Freitext \\ \hline
Entscheidungsdatum & Datum \\ \hline
Aktenzeichen & Freitext \\ \hline
Vorgericht & Freitext \\ \hline
Nachgericht & Freitext \\ \hline
Sachverhalt & Freitext \\ \hline
Gr�nde & Freitext \\ \hline
Unsichtbar & Bool \\ \hline
ausblenden & Bool \\ \hline
\end{tabular}
\label{Metadaten der Urteile in FADO}
\caption{Metadaten der Urteile in FADO}
\end{center}
\end{table}

\begin{table}[ht]
\begin{center}
\begin{tabular}{|l|l|}
\hline
\textbf{Relationen:} & \textbf{Werte:} \\ \hline
Thema & Vorgabe \\ \hline
geh�rt zu Fachobjekt & Vorgabe \\ \hline
hat Schlagwort & Vorgabe \\ \hline
ist vom Typ & Vorgabe \\ \hline
enthalten in Fachsystem & Vorgabe \\ \hline
wird referenziert von & Vorgabe \\ \hline
\end{tabular}
\label{Relationen der Urteile in FADO}
\caption{Relationen der Urteile in FADO}
\end{center}
\end{table}

\newpage
\subsubsection{Metadaten des Fachsystems FADO / Forschungsvorhaben}

\begin{table}[!ht]
\begin{center}
\begin{tabular}{|l|l|}
\hline
\textbf{Metadaten:} & \textbf{Werte:} \\ \hline
Fachsystem & Vorgabe \\ \hline
ID & Generiert \\ \hline
Title & Freitext \\ \hline
Kurzbeschreibung & Freitext \\ \hline
Kommentar & Freitext \\ \hline
F�rderbereich & Freitext \\ \hline
Beginn & Datum \\ \hline
Ende & Datum \\ \hline
Projektnummer & Freitext \\ \hline
F�rderkennzeichen & Freitext \\ \hline
Unsichtbar & Bool \\ \hline
ausblenden & Bool \\ \hline
\end{tabular}
\label{Metadaten der Forschungsvorhaben in FADO}
\caption{Metadaten der Forschungsvorhaben in FADO}
\end{center}
\end{table}

\begin{table}[!ht]
\begin{center}
\begin{tabular}{|l|l|}
\hline
\textbf{Relationen:} & \textbf{Werte:} \\ \hline
Thema & Vorgabe \\ \hline
geh�rt zu Fachobjekt & Vorgabe \\ \hline
hat Abschlussbericht & Vorgabe \\ \hline
hat Forschungsberichtsblatt & Vorgabe \\ \hline
hat Projektskizze & Vorgabe \\ \hline
hat Schlagwort & Vorgabe \\ \hline
hat Zwischenbericht & Vorgabe \\ \hline
wird geleitet von & Vorgabe \\ \hline
wird referenziert von & Vorgabe \\ \hline
\end{tabular}
\label{Relationen der Forschungsvorhaben in FADO}
\caption{Relationen der Forschungsvorhaben in FADO}
\end{center}
\end{table}

\newpage
\subsubsection{Metadaten des Fachsystems FADO / Berichte}

\begin{table}[!ht]
\begin{center}
\begin{tabular}{|l|l|}
\hline
\textbf{Metadaten:} & \textbf{Werte:} \\ \hline
Fachsystem & Vorgabe \\ \hline
ID & Generiert \\ \hline
Titel & Freitext \\ \hline
Kurzbeschreibung & Freitext \\ \hline
Kommentar & Freitext \\ \hline
Kurztitel & Freitext \\ \hline
Untertitel & Freitext \\ \hline
Fachthema & Freitext \\ \hline
Herausgeber & Freitext \\ \hline
Redaktion & Freitext \\ \hline
Version & Freitext \\ \hline
Stand & Datum \\ \hline
Seitenzahl & Freitext \\ \hline
Seite (von-bis) & Freitext \\ \hline
Reihe & Freitext \\ \hline
Bandnummer & Freitext \\ \hline
ISSN & Freitext \\ \hline
ISBN & Freitext \\ \hline
Preis & Freitext \\ \hline
Medium & Freitext \\ \hline
Shoprelevant & Bool \\ \hline
Shoplink & URL \\ \hline
HTML-Datei & Data \\ \hline
PDF-Datei & Data \\ \hline
Weitere Datei & Data \\ \hline
Format dieser Datei & Freitext \\ \hline
Unsichtbar & Bool \\ \hline
ausblenden & Bool \\ \hline
\end{tabular}
\label{Metadaten der Berichte in FADO}
\caption{Metadaten der Berichte in FADO}
\end{center}
\end{table}

\begin{table}[!ht]
\begin{center}
\begin{tabular}{|l|l|}
\hline
\textbf{Relationen:} & \textbf{Werte:} \\ \hline
betrift Thema & Vorgabe \\ \hline
geh�rt zu Fachobjekt & Vorgabe \\ \hline
hat Autor & Vorgabe \\ \hline
hat Schlagwort & Vorgabe \\ \hline
ist vom Typ & Vorgabe \\ \hline
enthalten in Fachsystem & Vorgabe \\ \hline
ist Abschlussbericht von & Vorgabe \\ \hline
ist Forschungsberichtsblatt von & Vorgabe \\ \hline
ist Projektskizze von & Vorgabe \\ \hline
ist Zwischenbericht von & Vorgabe \\ \hline
wird referenziert von & Vorgabe \\ \hline
\end{tabular}
\label{Relationen der Berichte in FADO}
\caption{Relationen der Berichte in FADO}
\end{center}
\end{table}

\newpage
\subsubsection{Metadaten des DRS}

\begin{table}[htbp]
\begin{center}
\begin{tabular}{|l|l|}
\hline
\textbf{Metadaten:} & \textbf{Werte:} \\ \hline
G�ltigkeit & Vorgabe \\ \hline
Titel & Freitext \\ \hline
Aktenzeichen & Freitext \\ \hline
Kurz-Titel & Vorgabe \\ \hline
Dokumentart & Vorgabe \\ \hline
Herausgeber & Vorgabe \\ \hline
Erscheinungsort & Vorgabe \\ \hline
Handbuch & Vorgabe \\ \hline
Kapitel & Vorgabe \\ \hline
Fundstelle & Vorgabe \\ \hline
Fassung & Vorgabe \\ \hline
�nderung & Vorgabe \\ \hline
Gr��e & Zahl \\ \hline
Formate & Vorgabe \\ \hline
\end{tabular}
\label{Metadaten des DRS}
\caption{Metadaten des DRS}
\end{center}
\end{table}

Zu beachten ist hier, dass die Metadaten-Tags Fassung und \"Anderung zusammengesetzte Daten enthalten, wie es in der Abbildung \ref{Suchmaske DRS} zu sehen ist.

\newpage
\subsubsection{Metadaten des Bildarchivs}

\begin{table}[htbp]
\begin{center}
\begin{tabular}{|l|l|}
\hline
\textbf{Metadaten:} & \textbf{Werte:} \\ \hline
Objektart & Freitext \\ \hline
Objektname & Freitext \\ \hline
ID & Generiert \\ \hline
URL & URL \\ \hline
Dateityp & Vorgabe \\ \hline
Name & Freitext \\ \hline
Kurzname & Freitext \\ \hline
Erstellt am & Datum \\ \hline
Autor & Freitext \\ \hline
Besitzer & Besitzer \\ \hline
Bemerkung & Freitext \\ \hline
\end{tabular}
\label{Metadaten des Bildarchivs}
\caption{Metadaten des}
\end{center}
\end{table}

\newpage
\subsection{Metadaten der ICT-ENSURE} \label{Metadaten der ICT-ENSURE}
In den folgenden Abschnitten sind die Metadaten der Dokumente der \ac{ICT-ENSURE} zu finden. Relationen im Datenmodell wurden hier durch die Mehrfachverwendung von Tags realisiert.

\subsubsection{Metadaten der Konferenzen}

\begin{table}[htbp]
\begin{center}
\begin{tabular}{|l|l|}
\hline
\textbf{Metadaten:} & \textbf{Werte:} \\ \hline
Editor & Freitext (Mehrere Autoren) \\ \hline
Publisher & Freitext (Bestehend aus Name und Ort) \\ \hline
Year of Pblishing & Jahr \\ \hline
ISBN & Freitext \\ \hline
Conferenc & Freitext (Bestehend aus Titel, Ort, Jahr) \\ \hline
\end{tabular}
\label{Metadaten der Konferenzen von ICT-ENSURE}
\caption{Metadaten der Konferenzen von ICT-ENSURE}
\end{center}
\end{table}

\subsubsection{Metadaten der Kapitel}

\begin{table}[htbp]
\begin{center}
\begin{tabular}{|l|l|}
\hline
\textbf{Metadaten:} & \textbf{Werte:} \\ \hline
Editor & Freitext (Mehrere Autoren) \\ \hline
Publisher & Freitext (Bestehend aus Name und Ort) \\ \hline
Year of Pblishing & Jahr \\ \hline
ISBN & Freitext \\ \hline
Conferenc & Freitext (Bestehend aus Titel, Ort, Jahr) \\ \hline
Editor & Freitext (Mehrere Autoren) \\ \hline
\end{tabular}
\label{Metadaten der Kapitel von ICT-ENSURE}
\caption{Metadaten der Kapitel von ICT-ENSURE}
\end{center}
\end{table}

\subsubsection{Metadaten der Vortr\"age}

\begin{table}[htbp]
\begin{center}
\begin{tabular}{|l|l|}
\hline
\textbf{Metadaten:} & \textbf{Werte:} \\ \hline
Konferenz & Vorgabe \\ \hline
Kapitel & Vorgabe \\ \hline
Author & Freitext (ggf. Mehrere) \\ \hline
Dateityp & Vorgabe \\ \hline
Titel & Freitext \\ \hline
Sprache & Freitext \\ \hline
Beginn Seite & Zahl \\ \hline
End Seite & Zahl \\ \hline
Schl�sselw�rter & Freitext \\ \hline
Abstract & Freitext \\ \hline
Volltext & Freitext \\ \hline
\end{tabular}
\label{Metadaten der Vortr\"age von ICT-ENSURE}
\caption{Metadaten der Vortr\"age von ICT-ENSURE}
\end{center}
\end{table}
