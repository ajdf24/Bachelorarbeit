\section{Analyse Datenbest\"ande} \label{Analyse Datenbestaende}
Im folgenden Kapiel werden die einzelnen Fachsysteme, der \ac{LUBW}, und deren Datenbest\"ande genauer analysiert. Welche Metdaten die einzelnen Dokumentenbest\"ande der Fachsysteme genau enthalten wird im Kapitel \ref{Stand der Technik} genauer erkl\"autert. 

Die einzelnen Fachsysteme der \ac{LUBW} sind f\"ur unterschieliche Einsatzzw\"acke entwickelt wurden, welche nun genauer anlaysiert werden.
\subsection{FADO und Untergruppen} \label{FADO}
Das \ac{FADO}-System der \ac{LUBW} erlaubt es den Nutzern nach verschiedenen Texte aus unterschielichen Themenrichtungen zu suchen und diese herunterzuladen. Hierbei stehen die Dokumente im PDF- oder HTML-Format bereit.

Die einzelnen Ver\"offentlichungen k\"onnen nach verschiedenen Kriterien durchsucht werden, wobei der Bestand an Dokumenten fortlaufend erweitert wird.
\cite{LUBW_FADO}

Im \ac{FADO}-System sind die Dokumente in die sechs Kategorien Altlasten, Boden, Natur und Landschaft, Umweltbeobachtung, Umweltforschung und Umweltinformationssysteme gegliedert. Hierbei ist zu beachten, dass ein Dokument nicht nur einem Themengebiet zugeordent werden kann, sondern durchaus unter mehreren Gebieten zu finden ist. Solche Dokumente sind im \ac{FADO} jedoch nur einmal abgespeichert und werden \"uber Relationen zu verschiedenen Kategorien zugeordent.

Die einzelnen Kategorien sind wiederum mit Untergruppen versehen, welche die Zugeh\"origkeit der Dokumente konkretisieren.

Da im Verlauf der Arbeit nicht alle Gruppen in ein neues System \"uberf\"uhrt werden k\"onnen, wird sich auf die Dokumentklassen Berichte, Urteile und Forschungsvorhaben beschr\"ankt, welche in der Kategorie Natur und Landschaft, in der Kategorie Boden beziehungsweise in der Kategorie Umweltforschung zu finden sind.

Forschungsvorhaben sind wiederum Berichte, welche sich in eine Skizze, einen Zwischen- und einen Abschlussbericht aufteilen. Hierbei geh\"oren die drei Teile zu jeweils einem Forschungsvorhaben.

\subsection{DRS} \label{DRS}
\subsection{Bilddatenbank} \label{Bilddatenbank}