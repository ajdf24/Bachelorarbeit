\section{Ausblick}
Da innerhalb der Arbeit weder das gesamte \ac{FADO}-System noch andere Systeme der \ac{LUBW} analysiert und in das Modell einbezogen wurden, m\"ussen alle noch fehlenden Dokumentenklassen mit ihren Metadaten zum Modell hinzugef\"ugt werden. Entsprechend muss auch das Datenmodell von Alfresco erweitert werden. Momentan fehlen im \ac{FADO}-System noch die Fachdokumente f\"ur "`Altlasten"', "`Chemikalien und Arbeitsschutz"', "`UIS Medien"', "`Umweltbeobachtung"' und "`Umweltforschung"', diese haben jedoch dasselbe Datenmodell wie die f�r "Boden", so dass es sich hier lediglich noch um weitere Export- und Importvorg�nge handelt.

Das Lastenheft verlangte, dass eine \"Anderung der grundlegenden Modells recht einfach gestaltet werden soll. Das Schreiben von XML-Quellcode ist jedoch f\"ur Laien nicht zumutbar, weshalb \"uberlegt werden sollte, ob die Erstellung eines Metadateneditors, welcher die XML-Dateien verwaltet, sinnvoll ist. Dieser Editor k�nnte gegebenenfalls auch die Konsistenz und Vollst�ndigkeit der erstellten bzw. bearbeiteten Dateien sicherstellen.

Das Alfresco-Formular zur Eingabe der Metadaten ist momentan, nach den einzelnen Aspekten und Typen gegliedert. Es werden ausnahmslos alle Metadaten angezeigt, egal ob sie f\"ur den entsprechenden Metadatensatz relevant sind oder nicht. Hier sollte in Absprache mit den jeweiligen Fachleuten der \ac{LUBW} Ordnung entstehen. Das hei\ss{}t, es m\"ussen optionale Metadaten ausgeblendet werden k\"onnen und die Gliederung der nach diesem Schritt verbleibenden Attribute sollte auf die Bed\"urfnisse der Mitarbeiter der \ac{LUBW} angepasst werden.

Die prototypische Implementierung des Backends ist bis auf die genannten Erweiterungen vollst\"andig.

Der gr\"o\ss{}ere Arbeitsaufwand wird bei der Erstellung eines Frontends entstehen. Wie in der Arbeit im Kapitel \ref{Implementierung Frontend} erw\"ahnt, war es nicht m\"oglich auf eine vollst�ndig implementierte Schnittstelle zwischen Alfresco und Liferay zur\"uckzugreifen. Daher muss entweder das bestehende CMIS-Repository erg�nzt oder eine Eigenl\"osung entwickelt werden.

In der Arbeit wurde prototypisch gezeigt, wie es m\"oglich ist, die Metadaten aus Alfresco �ber einen Hook in Liferay verf�gbar zu machen, Sollte dieser Ansatz in Zukunft weiter verfolgt werden, k\"onnen die gewonnenen Erkenntnisse und der entstandene Quellcode dieser Arbeit als Grundlage dienen. 

Es sollte im weiteren Verlauf des Projekts evaluiert werden, welche der in den Abschnitten \ref{Eingriff in den Liferay Quellcode} bis \ref{Liferay 7} diskutierten Alternativen am besten f\"ur einen sp\"ateren produktiven Einsatz geeignet ist.

