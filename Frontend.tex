\section{Implementierung des Frontends auf Basis von Liferay} \label{Implementierung Frontend}
Liferay ist ein hochperformanter Portalserver, welcher wie auch Alfresco Open Source ist. Der Nutzer hat bei Liferay die Auswahl zwischen einer kostenlosen Community Edition und einer kostenpflichtigen Enterprise Edition. Vorteil der kostenpflichtigen Enterprise Edition ist, dass hier ein Support zur Verf\"ugung steht. Au\ss{}erdem werden die kostenpflichtigen Versionen vor einer Ver\"offentlichung genauer getestet, als die Community Edition. \cite{Alfresco_und_Liferay} \cite{Wiki_Liferay}

Liferay nutzt f\"ur die Darstellung von Content sogenannte "`Portlets"'. Diese Portlets sind Anwendungen, welche Nutzern im Web zur Verf\"ungung gestellt werden. Dabei k�nnen mehrere Portlets zu dynamischen (Web-)Seiten zusammengestellt werden. Jedes Portlet besteht daher aus einer clientseitigen Anzeige, kann jedoch auch auf Serverseite Anwendungslogik enthalten, die zum Beispiel den Zugriff auf Daten regelt.

\subsection{Alfresco als IFrame im Liferay}
In Liferay ist es m\"oglich, einen Webcontent, wie das Alfresco-Portal einer ist, als IFrame darzusellen. Der Vorteil einer solchen Umsetzung ist, dass sie schnell und einfach \"uber ein Web-Content-Portlet umgesetzt werden kann. Ein gro\ss{}er Nachteil eines solchen IFrames ist, dass  es sich nur um ineinander geschachtelte Ansichten von getrennten Systemen handelt. Darstellung, Daten und Logik bleiben - wie auch die Benutzerverwaltung - komplett getrennt. Eine einheitliche Anwendung l�sst sich auf diese Weise nicht gestalten. Daher kann die Einbindung per IFrame bestenfalls als Notl�sung dienen.

\subsection{Liferay-Repository als Frontend f\"ur Alfresco Dateien}
Auch Liferay selbst bietet grundlegend die M\"oglichkeit, eines Asset- und Dokumentenmanagementsystes. Zus\"atzlich k\"onnen auch Dateien aus anderen Systemen als Repository zu Liferay hinzugef\"ugt werden. Sie werden dann so behandelt, als seien sie Liferay-eigene Objekte. Tats�chlich werden "`echte"', redundante Liferay-Objekte erstellt, die automatisch mit der Quelle synchronisiert werden. Der Vorteil dieses Ansatzes ist, dass �ber diese Repository Objekte aus fremden Datenquellen �ber dieselben Mechanismen behandelt werden k�nnen wie die Liferay-eigenen. So k\"onnen diese Dateien, zumindest konzeptionell, zum Beispiel �ber Standard-Liferay-Portlets gefiltert, sortiert und angezeigt werden.

Das Hinzuf\"ugen eines Repositorys ist im Grunde ganz einfach, vorausgesetzt die Einstellungen der Benutzerverwaltung sind entsprechend gesetzt, denn um auf ein Repository von Alfresco zugreifen zu k\"onnen, muss sich der Benutzer authentifizieren. (siehe Abschnitt \ref{Benutzerverwaltung f\"ur eine Repositoryzugriff konfigurieren})

\subsubsection{Benutzerverwaltung f\"ur eine Repositoryzugriff konfigurieren}\label{Benutzerverwaltung f\"ur eine Repositoryzugriff konfigurieren}
Das im Abschnitt \ref{Alfresco als Repository im Liferay} verwendete CMIS-Repository funktioniert nur, wenn zuvor eine Authentifizierung des Nutzers stattgefunden hat. Normalerweise wird so eine Benutzerverwaltung von einem LDAP-Server oder \"uber SSO vorgenommen. 

Es reicht f\"ur die prototypische Implementierung in dieser Arbeit aber vollkommen aus, den gleichen Benutzer in beiden Systemen zu erstellen. Hierbei m\"ussen die Passw\"orter und die Benutzernamen in beiden Systemen \"ubereinstimmen.

Ist der selbe Nutzer in beiden Systemen erstellt, muss in die Liferay-Datei \\\texttt{portal-setup-wizard.properties} welche im Order \texttt{LIFERAY\_HOME} liegt, noch die im Listing \ref{Datei Erweiterung portal-setup-wizard.properties} zu sehende Zeile hinzugef\"ugt werden. Dies erlaubt es Liferay, das Passwort innerhalb einer Sitzung zu speichern und es f\"ur die \ac{CMIS}-Verbindung zu nutzen. \cite{CMIS_Repo}

\lstinputlisting[caption=Erweiterung in der Datei portal-setup-wizard.properties, label=Datei Erweiterung portal-setup-wizard.properties]{Code/portal-setup-wizard.properties}

Zus\"atzlich muss in Alfresco unter \texttt{Admin \MVRightarrow Kontrollbereich \MVRightarrow Portaleinstellungen \MVRightarrow} \\
\texttt{Authentifizierung} im Feld \texttt{Wie authentifizieren sich Nutzer?} "`Mit Benutzername"' angegeben werden.

Nach einem Neustart des Servers muss sich ab sofort mit dem Benutzernamen angemeldet werden und nicht wie zuvor mit der Mail-Adresse. \cite{CMIS_Config}

\subsubsection{Alfresco als Repository im Liferay}\label{Alfresco als Repository im Liferay}
In Liferay kann auf der Seite "`Inhalte"' unter "`Dokumente und Medien"' ein neues Repository erzeugt werden. Es muss ein Name f\"ur das Repository gew\"ahlt werden, unter dem dieses sp\"ater zu finden sein soll. Als Repositorytyp muss ein \ac{CMIS}-Repository des Typs "`AtomPub"' gew\"ahlt werden. 

Die passende URL von Alfresco kann auf der Seite \url{http://127.0.0.1:8080/alfresco/} gefunden werden. Es k\"onnen die AtomPub Version 1.0 oder 1.1 gew\"ahlt werden. Zus\"atzlich finden sich hier auch weitere Links, wie zum Beispiel auf die WebDav-Schnittstelle von Alfresco.

Ist die richtige URL im Feld "`AtomPub-URL"' eingetragen, muss noch die Berechtigung gew\"ahlt und anschlie\ss{}end gespeichert werden. Die Felder "`Beschreibung"' und "`Repository ID"' k\"onnen frei bleiben.

Ist alles richtig konfiguriert, sind nun alle Dateien aus Alfresco auch in Liferay abrufbar (siehe Abbildung \ref{Repositorydarstellung in Liferay Bild} im Anhang \ref{Repositorydarstellung in Liferay}) und k\"onnen innerhalb von Liferay verwendet und dargestellt werden. \cite{CMIS_Repo}

Jedoch stellt Liferay nur Metadaten dar, welche in der Datei selbst vorhanden sind. Alle anderen Metadaten, welche in Alfresco vorhanden sind, werden ignoriert. Im folgenden Abschnitt \ref{Untersuchungen zum Liferay-Repository} wird genauer untersucht, warum die Alfresco-Metadaten nicht dargestellt werden. \cite{Intregrate_as_Repo} 

\subsubsection{Untersuchungen zum Liferay-Repository}\label{Untersuchungen zum Liferay-Repository}
Zun\"achst muss \"uberpr\"uft werden, ob Alfresco \"uberhaupt alle vorhandenen Metadaten �ber die CMIS-Schnittstelle zur Verf�gung stellt.

\"Uber den Link im Listing \ref{Abfrage mit CMIS} k\"onnen alle Informationen zu einem Dokument (Node) anhand seiner ID abgefragt werden. \cite{GetNodeInfo}

\lstinputlisting[language=html, caption=Abfrage aller Propertys eines Nodes mittels CMIS 1.1, label=Abfrage mit CMIS]{Code/getNodeInformation.html}

Beim Aufrufen des Links muss zuerst eine Authentifizierung f\"ur Alfresco angeben werden. War diese korrekt, so wird im Browser das entsprechende Element mit allen Metadaten in XML dargestellt. Dies ist zum einen die Best\"atigung, dass der \ac{CMIS}-Dienst von Alfresco ohne Probleme l\"auft, aber zum anderen auch, dass Alfresco alle vorhandenen Metadaten \"ubermittelt.

Das Problem liegt also nicht auf der Seite von Alfresco und es muss untersucht werden, ob die Daten im Liferay-Server richtig ankommen und verarbeitet werden.

Um zu verifizieren, ob Liferay die Metadaten von Alfresco gelesen hat, muss in die Datenbank von Liferay geschaut werden.

Liferay nutzt zur internen Datenspeicherung standardm\"a\ss{}ig eine HyperSQL-Datenbank, welche vollst\"andig in Java implementiert ist und unter der BSD-Lizenz steht. Um die Datenbank auszulesen, wird ein Tool ben\"otigt, welches jedoch mit dem Liferay-Paket mitgeliefert wird. Die jar-Datei befindet ist unter \texttt{LIFERAY\_HOME/tomcat-7.0.42/lib/ext/hsql.jar}.

Nach dem Start muss als URL \url{jdbc:hsqldb:<Pfad zum Liferay-Home>/LIFERAY\_HOME/data/hsql/lportal} angegeben werden um sich mit der Datenbank zu verbinden. Vorher muss der Liferay-Server heruntergefahren werden, da dieser ein Lock auf die Datenbank h\"alt und sie somit sperrt.

Liferay kann aber auch mit einer anderen Datenbank, wie zum Beispiel MySQL, eingerichtet werden. Voraussetzung hierf\"ur ist, dass der Datendank-Server dann auf dem System vorhanden ist.

Innerhalb der Datenbank sind in der Tabelle \texttt{PUBLIC.DDMCONTENT} auch die Metadaten der Dateien abgelegt, welche \"uber das Repository eingef\"ugt wurden. In der Spalte \texttt{XML} werden alle Metadaten zu einem Dokument abgelegt. Hier f\"allt auf, dass nicht alle Metadaten, welche Alfresco bietet, angegeben sind. Lediglich standardisierte Metadaten, welche aus der Datei ausgelesen wurden, sind hier zu finden. 

Daher ist es nicht verwunderlich, dass Liferay keine weiteren Metadaten anzeigt, denn sie sind einfach nicht bekannt. Offensichtlich werden die Alfresco-Metadaten �ber die CMIS-Schnittstelle von Liferay nicht vollst�ndig �bernommen und auf entsprechende Liferay-Metadatenfelder abgebildet.

Im Folgenden muss nun gepr\"uft werden, wie die Metadaten in Liferay bekannt gemacht werden k\"onnen. Hierbei m\"ussen die Metadaten aus Alfresco ausgelesen, in Liferay gespeichert und angezeigt werden.

Nach einiger recherche stellte sich heraus, dass Liferay nicht in der Lage ist eigene Metadatenmodelle aus Alfresco zu verarbeiten, da diese nicht standardisiert sind. Im Abschnitt \ref{Eingriff in den Liferay Quellcode} wird hierauf noch einmal eingegangen. \cite{Liferay_CMIS_Forum} 

\subsection{Alternativen zum bestehenden Liferay-Repository}
In den vorherigen Kapiteln wurde beschrieben, wie es m\"oglich ist, Alfresco-Daten in Liferay zu integrieren. Es wurde jedoch festgestellt, das Liferay standardm\"a\ss{}ig nicht in der Lage ist, die Metadaten von Alfresco-Dokumenten vollst\"andig zu verarbeiten oder anzuzeigen. Daher wird im Folgenden nun auf Alternativen eingegangen.

Grunds\"atzlich gibt es zwei m\"ogliche Alternativen zum Liferay-\ac{CMIS}-Repository. Zum einen besteht die M\"oglichkeit ein eigenes, unabh\"angiges Portlet, einen Hook oder eine andere Technologie f\"ur Liferay zu entwickeln, zum anderen ist es m\"oglich, in den bestehenden Quellcode von Liferay einzugreifen. Die M\"oglichkeit eines Eingriffs in den Quellcode ist grunds\"atzlich m\"oglich, da es sich, wie schon erw\"ahnt, bei Liferay um ein Open-Source-Projekt handelt. Es w\"are somit m\"oglich, die \ac{CMIS} Schnittstelle von Liferay zu erweitern, so dass sie die von Alfresco gegebenen Metadaten verwalten kann. Nat\"urlich gibt es noch andere M\"oglichkeiten, welche in den Folgenden Abschnitten beschrieben werden.

\subsubsection{Eingriff in den Liferay Quellcode}\label{Eingriff in den Liferay Quellcode}
Im Liferay Quellcode\footnote{\url{https://github.com/liferay/liferay-portal}} konnten mehrere Stellen identifiziert werden, wo die Apache-Chemistry Bibliothek, welche den Open-\ac{CMIS}-Standard implementiert, verwendet wird. N\"ahere Infos hierzu sind im n\"achsten Abschnitt zu finden.

Die Verwendung dieser Bibliothek deutet eindeutig auf die Implementierung von \ac{CMIS} hin. Im Quellcode konnten somit mehrere Klassen gefunden werden, welche f\"ur die Verbindung mit Repositorys zust\"andig sind. So auch die Klasse \texttt{CMISRepository.java}\footnote{\url{https://github.com/liferay/liferay-portal/blob/a3529628fb7bab5f22d4885ea32cce50703b995f/modules/apps/document-library/document-library-repository-cmis-impl/src/com/liferay/document/library/repository/cmis/internal/CMISRepository.java}} welche die Abfrage von Dokumenten \"ubernimmt. An dieser Stelle m\"usste in den Quellcode eingegriffen werden, damit Metadaten von Alfresco nicht nur geladen, sondern auch in Liferay gespeichert  werden k\"onnen.

Damit jedoch nicht genug, es muss auch geschaut werden wie die Speicherung und die Anzeige der gelandenen Metadaten umgesetzt werden kann. Hierf\"ur war jedoch im Rahmen der Arbeit nicht gen\"ugent Zeit, weshalb dieser Ansatz nicht weiter verfolgt wurde.

\subsubsection{Entwicklung eines neuen Liferay-Hooks}\label{Liferay Hook}
Eine M\"oglichkeit, die fehlenden Metadaten aus Alfresco zu laden, w\"are \"uber einen Hook. Dieser sogenannte Hook ist ein Hintergrundprozess, der ohne grafische Schnittstelle auskommt. \cite{Liferay_in_Action}

Um \"uberhaupt eine Erweiterung f\"ur Liferay schreiben zu k\"onnen, wird die Liferay IDE\footnote{\url{http://sourceforge.net/projects/lportal/files/Liferay\%20IDE/}} ben\"otigt. Hierbei handelt es sich um eine modifizierte Eclipse-Plattform, welche einen Nutzer in die Lage versetzt, Komponenten f\"ur Liferay zu entwickeln.


Die Klasse des Hooks implementiert das Interface \texttt{ModelListener}, welcher auf \"Anderungen der Klasse \texttt{RepositoryEntry} achtet.
In der implementierten Methode \texttt{onAfterCreate()} werden dann die Metadaten von Alfresco \"uber die \ac{CMIS}-Schnittstelle abgeholt und in die Liferay-Datenbank gespeichert. Die Methode wird aufgerufen, sobald ein Element aus dem Repository gelesen und in der Liferay-Datenbank angelegt wird. Dies geschieht zum Zeitpunkt der ersten Nutzung. Die Synchronisation von Daten muss sp\"ater der Hook selbst \"ubernehmen, denn automatisch werden nur neue oder ge\"anderte Daten geladen.

F\"ur das Abrufen der Alfresco-\ac{CMIS}-Schnittstelle wurde die Apache Bibliothek "`Chemistry"'\footnote{\url{https://chemistry.apache.org/}} verwendet, welche es erm\"oglicht, \ac{CMIS}-Schnittstellen anzusprechen.

Die genaue Implementierung ist im Listing \ref{CMIS mit Chemistry} zu sehen. Es wird zuerst eine Session mit den ben\"otigten Parametern wie Username, Password und AtomPub-URL erzeugt. Dies geschieht in der selbst erstellten Methode \texttt{createCmisSession()}, welche im Anhang \ref{Methode createCmisSession} zu sehen ist.

Anschlie\ss{}end wird f\"ur das gerade erstellte Element die ID geholt und mit ihrer Hilfe das \texttt{CmisObjekt} erstellt.
Handelt es sich um einen Ordner, wird nichts weiter unternommen. Ist das Element jedoch ein Dokument, so werden alle Propertys geholt.

\"Uber einen Iterator wird durch alle Propertys durchgegangen und diese mit Hilfe der \\\texttt{ExpandoBridge} in die Liferay-Datenbank eingebracht.
\newpage
\lstinputlisting[language=java, caption=Prototypische-Implementierung einer \ac{CMIS}-Schnittstelle mit Apache Chemistry, label=CMIS mit Chemistry]{Code/Chemistry.java}

Dieser kleine Ausflug beweist, dass es m\"oglich ist, die von Alfresco ankommenden Metadaten in Liferay zu integrieren. Somit steht auch dem Anzeigen der Metadaten auf der Oberfl\"ache nichts mehr im Wege. \cite{Chemistry_examples}

Der Aufwand, eine komplette Portierung der Metadaten umzusetzen ist relativ gering, zumal der im Prototyp vorhandene Code ohne gr\"o\ss{}ere \"Anderungen \"ubernommen werden kann. Es muss lediglich noch ein Filter eingebaut werden, um nur die gew\"unschten Metadaten in der Datenbank zu speichern.

Die Umsetzung der Anzeige und einer Suchfunktion in dem der "`Asset Publisher"' erweitert wird, sollte bei vorhandenen Daten ohne Probleme m\"oglich sein.

Zu beachten ist, dass der erstellte Hook Daten nur einmal l\"adt. F\"ur die sp\"atere Synchronisation sind weitere Implentierungsschritte notwendig.

\subsection{Entwicklung eines neuen Liferay-Portlets}\label{Liferay Portlet}
Im Abschnitt \ref{Liferay Hook} wurde beschrieben, wie die Metadaten von Alfresco abgefragt und in der Datenbank gespeichert werden k\"onnen.

Eine andere  M\"oglichkeit, ist die Daten nicht \"uber einen Hintergrundprozess abzufragen, sondern direkt in einem Portlet. Dies hat den Vorteil, dass eine Filterung der Metadaten nach gewissen Kriterien von Alfresco vorgenommen wird. Auch entsteht durch die Live-Abfrage keine Datenduplikation, was jedoch vermutlich die Abfragegeschwindigkeit verlangsamt.
Zus\"atzlich muss f\"ur ein Portlet auch eine eigene Oberfl\"ache implementiert werde, welche die abgefragten Daten von Alfresco anzeigt.

Ob sich daher die Implementierung eines Portlets vom Aufwand und der Abfragegeschindigkeit her lohnt kann leider nicht genau gesagt werden. Dies soll aber im Verlauf der weiteren Entwicklung evaluiert werden.

\subsubsection{Entwicklung von Widgets}\label{Liferay Widget}
Innerhalb der Projektgruppe werden oft kleine JavaScript-Widgets verwendet, die gewisse Informationen darstellen und auch Abfragen k\"onnen. 
Diese JavaScript-Widgets lassen sich beliebig in jede Webseite und auch in Liferay integrieren und k\"onnen dort Verwendung finden.F�r die m�glichst nahtlose Integration in Liferay k�nnen Widgets in Portlets verpackt werden, die f�r Redakteure und Autoren dann wie herk�mmliche, native Portlets funktionieren.

Es m\"usste also ein Widget geschrieben werden, welches die \ac{CMIS}-Daten von Alfresco auslie\ss{}t und anzeigt. Zus\"atzlich muss es auch eine Suchfunktion bieten, welche es erlaubt die Dokumente nach bestimmten Metadaten zu filtern. Der Arbeitsaufwand sollte in etwa \"ahnlich dem eines Portlets (siehe \ref{Liferay Portlet}) sein, jedoch mit dem Vorteil, dass ein solches Widget \"uberall integriert werden kann, wo JavaScript lauff\"ahig ist. 

F\"ur die Anbindung an Alfresco kann direkt die Alfresco-JavaScript-\ac{API} verwendet werden.

Eine weitere, unabh�ngige M�glichkeit w�re die Synchronisation der Alfresco-Metadaten mit einer strukturierten Suchmaschine wie zum Beispiel ElasticSearch\footnote{https://www.elastic.co/products/elasticsearch}. Diese stellt zwar zun�chst nur eine andere Schnittstelle mit denselben Daten zur Verf�gung, jedoch existieren bereit Widgets zur Abfrage, Filterung und Darstellung von ElasticSearch-Daten. Mit Alfresco Acitiviti steht ein Plugin zur Synchronisation von Alfresco und ElasticSearch zur Verf�gung. Eine Umsetzung auf dieser Basis wurde jedoch im Rahmen dieser Arbeit aus Zeitgr�nden nicht gemacht. \cite{Wiki_Elastic} \cite{elastic}

\subsubsection{Liferay Enterprise Edition}\label{Liferay EE}
Innerhalb der Arbeit wurde auch getestet, ob die Enterprise Edition von Liferay im Gegensatz zur bisher verwendeten Community Edition in der Lage ist, alle \ac{CMIS}-Daten abzufragen und zu speichern. Hierf\"ur wurde die Testversion genutzt, welche zwar den vollen Funktionsumfang bietet, jedoch auf 30 Tage limitiert ist.

Es wurde kein Unterschied zum Verhalten der Community Edition festgestellt. Auch bietet der Store von Liferay in der Enterprise Edition keine anderen Alfresco- oder \ac{CMIS}-Plugins als in der Community Edition. Somit bietet die Verwendung der Enterprise Edition f\"ur die vorliegende Aufgabe keine entscheidenden Vorteile.

\subsubsection{Verwendung von Liferay 7}\label{Liferay 7}
In Liferay 7, welches sich zur Zeit noch in der Entwicklung befindet, wird zum einen das Dokumentmanagement von Liferay umgebaut und zum anderen die \ac{CMIS}-Schnittstelle erweitert. \cite{liferay7}

Es ist somit gut m\"oglich, dass Liferay sehr bald schon eigenst\"andig in der Lage ist, alle Metadaten von Alfresco anzuzeigen und zu verwalten.
Im Verlauf der Arbeit wurde eine "`Mile-Stone"'-Version installiert, um die Behauptung genauer zu pr\"ufen.

Es konnte in der Vorabversion von Liferay zwar ein Repository angelegt werden, jedoch war die Version nicht in der Lage, Dokumente zu laden. Die Logausgabe von Liferay legt nahe, dass momentan die Datenbank-Querys noch fehlerhaft sind. Daher konnte die Fuktionalit\"at der neuen \ac{CMIS}-Schnittstelle nicht \"uberpr\"uft werden.

\subsection{Auswertung der M\"oglichkeiten}
Es ist ohne weiteres Zutun nicht m\"oglich, die Metadaten aus Alfresco direkt in Liferay anzuzeigen. Es wurde aber im Abschnitt \ref{Liferay Hook} gezeigt, dass eine nachtr\"agliche Implementierung dieser Funktionalit\"at grunds\"atzlich m\"oglich ist. Aus zeitlichen Gr\"unden war es aber im Rahmen dieser Arbeit nicht machbar gewesen, ein vollst\"andiges Beispiel zu Implementieren.

Andere Alternativen wie die Entwicklung von Widgets, Portles oder Hooks sind grunds\"atzlich m\"oglich. Eine Implementierung ist zwar m\"oglich, jedoch aufw\"andig, weshalb dieser Ansatz nur prototypisch im Abschnitt \ref{Liferay Hook} beschrieben wurde. Die nur lose Kopplung von Alfresco und Liferay-Daten �ber Widgets und Portlets erschwert die Verkn�pfung von Daten aus beiden Welten und damit die Erstellung von �bergreifenden Anwendungen.

Neben der Erweiterung des CMIS-Repositories, im Liferay-Code, erscheint als bisher die beste L\"osung die Erweiterung von Liferay \"uber einen Hook, da es f\"ur Liferay keine fertigen Plugins gibt. Eine weitere M\"oglichkeit besteht darin, abzuwarten was Liferay in der Version 7 bietet. Um die Live-Abfrage von Portlets oder Widgets zu beschleunigen, k\"onnten die Alfresco-Daten zum Beispiel in eine ElasticSearch Suchmaschine eingebracht und \"uber diese abgefragt werden. 
