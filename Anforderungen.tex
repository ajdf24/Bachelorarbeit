\section{Lastenheft} \label{Lastenheft}
An das Projekt, welches im Rahmen dieser Arbeit bearbeitet werden soll, gibt es viele Anforderungen. Um alle Anforderungen m\"oglichst strukturiert und \"Ubersichtlich darzustellen, wurde die Form eines Lastenhefts gew\"ahlt.

\subsection{Zielsetzung} \label{Zielsetzung}
Ziel ist es ein einheitliches \ac{ECM} System zu entwerfen, welches m\"oglichst alle Fachdokumente der \ac{LUBW}, der \ac{ICT-ENSURE} und der \ac{GAA} beinhaltet.
Hierbei sollen alle bestehenden und zuk\"unfitgen Systeme \"uber eine Schnittstelle mit dem \ac{DMS} kommunizieren.

\subsection{Produkteinsatz} \label{Produkteinsatz}
Bei dem zu entwickelnden System, handelt es sich um einen Prototyp f\"ur ein \ac{DMS}, welches gegebenenfalls von der \ac{LUBW} und der \ac{GAA} eingesetzt werden kann. Das zu entwickelnde Backend, soll alle Dokumente, welche schon heute auf den Websiten zu finden sind zusammenf\"uhren und diese in einer geeigneten Form ablegen.

\subsubsection{Zusammenspiel mit anderen Systemen} \label{Zusammenspiel mit anderen Systemen}
Das Backend soll nicht die heutigen Insell\"osungen f\"ur Dokumentverwaltung der \ac{LUBW} und der \ac{GAA} vereinen. Es sollen sowohl die bestehenden als auch zuk\"unfitge Systeme unterst\"utzt werden. Die jeweils ben\"otigten Daten, sollen hierf\"ur \"uber eine oder gegebenenfalls mehrere Schnittstellen erreichbar sein.

F\"ur die \ac{ICT-ENSURE} soll ein eigenes System entwickelt werden, welches jedoch auf der gleichen Datenbasis aufbaut wie das der \ac{LUBW}.

\subsection{Produktfunktionen} \label{Produktfunktionen}
\begin{minipage}{3cm}
/LF10/
\end{minipage}
\begin{minipage}{13cm}
Schon bestehende Dokumente sollen aus den alten Systemen \"ubernommen werden k\"onnen, ohne das ihre Metadaten verloren gehen oder ge\"andert werden. Dies bedeutet, dass das neue System den schon vorhandenen Datenbestand abbilden muss.\\
\end{minipage}
\begin{minipage}{3cm}
/LF20/
\end{minipage}
\begin{minipage}{13cm}
Die Metadaten der einzelnen Dokumente sollen fachlich und technisch gegliedert werden. Hierbei sollen gegebenenfalls Gruppierungen erstellt werden. Zum Beispiel sollen Longitude und Latitude zur Metadaten-Gruppe "`Geodata"' zusammengefasst werden.\\
\end{minipage}
\begin{minipage}{3cm}
/LF30/
\end{minipage}
\begin{minipage}{13cm}
Die Dokumente sollen ebenfalls nach den einzelnen Anwendungen und Fachbereichen gruppiert werden. Hier sollen die Anwendungen auch speziell ihre Dokumente nach Gruppenzugeh\"origkeit abfragen k\"onnen. Dies ist wichtig, da im \ac{FADO} nur die entsprechenden \ac{FADO}-Dokumente angezeigt werden sollen.\\
\end{minipage}
\begin{minipage}{3cm}
/LF40/
\end{minipage}
\begin{minipage}{13cm}
Eine Suche \"uber die Dokumente und ihre Metadaten soll grunds\"atzlich m\"oglich sein. Hier muss der Nutzer aber auch in der Lage sein die Suche weiter einzuschr\"anken oder nur nach bestimmten Metadaten suchen zu k\"onnen.\\
\end{minipage}
\begin{minipage}{3cm}
/LF50/
\end{minipage}
\begin{minipage}{13cm}
Die alten Systeme sollen durch das neue Backend keinerlei Einschr\"ankungen im Funktionsumfang unterworfen sein. Dies muss gegebenenfalls durch verschiedene Arten von Schnittstellen realisert werden.\\
\end{minipage}
\begin{minipage}{3cm}
/LF60/
\end{minipage}
\begin{minipage}{13cm}
F\"ur den Frontend-Prototyp, soll die Kernfunktionalit\"at einer Schnittstelle funktionsf\"ahig sein, anhand dessen der Prototyp evaluiert werden kann.\\
\end{minipage}

\subsection{Produktdaten} \label{Produktdaten}
\begin{minipage}{3cm}
/LD10/
\end{minipage}
\begin{minipage}{13cm}
Die entsprechenden Metadaten zu Dokumenten sollen durch das Backend verwaltet werden. Hierbei sollen nicht nur die manuell hinzugef\"ugten Metadaten beachtet werden, sondern auch die Metadaten, welche das Dateiformat liefert und solche die das Dokument enth\"alt.\\
\end{minipage}
\begin{minipage}{3cm}
/LD20/
\end{minipage}
\begin{minipage}{13cm}
Metadaten sollen nach M\"oglichkeit so zusammengefasst werden, das eine Oberklassenbildung und Vererbung von Metadaten m\"oglich ist. So soll zum Beispiel ein Standort alle Adressdaten zusammenfassen. Ein Standort wiederum kann in mehreren Dokumenten verwendet werden, wie zum Beispiel in Bildern oder in einem Forschungsbericht. Wiederum k\"onnte der Standort aber auch Betandteil der Unterklasse Gericht sein und somit den Standort des Gerichts abbilden.\\
\end{minipage}
\begin{minipage}{3cm}
/LD30/
\end{minipage}
\begin{minipage}{13cm}
Metadaten sollen \"ahnlich wie objektorientierte Klassen in Programmiersprachen agieren. So wird ein Autor "`Max Mustermann"' nur einmal angelegt. Er kann jedoch in mehreren Dokumenten auftauchen. Sucht ein Nutzer nach "`Max Mustermann"', so werden ihm alle Eintr\"age zu diesem angezeigt. Vergleichen kann mal diese Art von Abbildung auch mit einer relationalen Abbildung wie sie in Datenbanken zu finden ist.\\
\end{minipage}
\begin{minipage}{3cm}
/LD40/
\end{minipage}
\begin{minipage}{13cm}
F\"ur die Gliederung der Metdaten sollen auch schon bestehende Standards betrachtet und gegebenenfalls umgesetzt werden.
Im Verlauf der Arbeit muss evaluiert werden, welche Standards es f\"ur Metadaten gibt und wie sie im Projekt umgesetzt werden m\"ussen und k\"onnen.
\cite{Wiki_Dublin_Core} \cite{Wiki_ISO_19115} \cite{Wiki_Exif} \cite{Wiki_Inspire}\\
\end{minipage}


\subsection{Produktleistungen} \label{Produktleistungen}
\begin{minipage}{3cm}
/LL10/
\end{minipage}
\begin{minipage}{13cm}
Das Backend soll f\"ur Dokumente, welche in verschiedenen Versionen vorliegen eine Art Versionsverwaltung bieten, so das auch \"altere Versionen zug\"anglich sind. Dies zum Beispiel bei Gesetzestexten wichtig, da hier immer die zum Vorfall aktuelle Version betrachtet werden muss.\\
\end{minipage}
\begin{minipage}{3cm}
/LL20/
\end{minipage}
\begin{minipage}{13cm}
Die Dokumente sollen \"uber verschiedene Schnittstellen wie zum Beispiel eine \ac{REST}-Schnittstelle abgerufen werden k\"onnen. Der Einsatz von \ac{API}s ist je nach \ac{ECM}-Tool gegebenenfalls auch m\"oglich, jedoch ist eine solche Umsetzung nicht Teil der Arbeit. \cite{Wiki_REST}\\
\end{minipage}
\begin{minipage}{3cm}
/LL30/
\end{minipage}
\begin{minipage}{13cm}
Eine Suche \"uber Metadaten muss realisert werden. Hierbei soll dem Nutzer zum einen eine grobe aber auch zum anderen eine verfeinerte eingeschr\"ankte Suche nach bestimmten Metadaten geboten werden.\\
\end{minipage}