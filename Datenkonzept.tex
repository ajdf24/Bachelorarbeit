\section{Erstellung eines Datenkonzepts} \label{Erstellung eines Datenkonzepts}
Im nun folgenden Kapitel werden die zuvor in den Kapiteln \ref{Analyse Datenbestaende} und \ref{Stand der Technik} beschriebenen Metadaten der einzelnen Systeme zu einen umfassenden Datenkonzepts zusammengefasst. Dieses Konzept bildest die Grundlage zur Speicherung in einem \ac{DMS}.

F\"ur die grafische Darstellung des Metadatenmodells wurde eine UML-Darstellung gew\"ahlt, wobei Klasse eine Datensammlung dartstellt. 

Es wird explizit darauf hingewiesen, dass die UML-Darstellung kein Klassendiagramm im eigentlichen Sinn ist und somit von der von der vorgeschriebenen Darstellungsform abgewichen wurde.

Attribute, welche sich direkt in der Sammlung befinden sind ohne besondere Kennzeichnung einfach dargestellt. Andere Attribute, welche wiederum auf eine Datensammlung verweisen, sind mit dem jeweiligen Verweistyp gekennzeichnet. Zus\"atzlich wird mit Hilfe der Hintergrundfarbe sichtbar gemacht, in welchem Package sich die Datensammlung befindet. 

Verweise sind zus\"atzlich \"uber Pfeile gekennzeichnet, an welchen die Kardinalit\"at zu finden ist.

Um in der Arbeit eine \"Ubersicht zu geben, werden die Packages einzeln beschriebenen. Eine gesamte \"Ubersicht des Diagramms ist auf dem der Arbeit beiliegenden Datentr\"ager zu finden.


\subsection{FADO Metadaten}
\subsection{DRS Metadaten}
\subsection{Bildarchiv Metadaten}
\subsection{ICT-ENSURE Metadaten}

\subsection{Untergeordnete Datensammlungen}
\subsubsection{Gerichtbarkeit}
\subsubsection{Bibliographie Metadaten}
\subsubsection{Abstrakte Metadaten}
\subsubsection{Standard Metadaten}
\subsubsection{Abstrakte Standard Metadaten}
\subsubsection{Grundlegenden Daten}
\subsubsection{Urheber Metadaten}
\subsection{Metadatenmapping im neuen Modell}
Da sich durch das zusammengefassen und dem aufstellen eines \"ubergreifenden Modells der Metadaten einige Bezeichnungen der Attribute ge\"andert haben, wird in diesem Abschnitt nun das Mapping zwischen dem alten und neuen Modell beschrieben. Hierf\"ur werden die Metadaten der einzelnen Systeme aufgeschl\"usselt.

\subsubsection{FADO}
\subsubsection{DRS}
\subsubsection{Bildarchiv}
\subsubsection{ICT-ENSURE}