\section{Einleitung} \label{Einfuehrung}
Beh\"orden und Firmen haben heute immer gr\"o\ss{}ere Datenbest\"ande zu verwalten, welche schon in elektronischen Systemen vorzufinden sind. Da diese Systeme historisch bedingt gewachsen sind, entstanden im Laufe der Zeit immer mehr kleinere Insell\"osungen, welche ein \"ubergreifendes Verwalten und Nutzen von Daten erschwert oder gar ganz verhindert.

Das Ziel dieser Arbeit ist daher, ein Konzept sowie eine prototypische Implementierung zu erstellen, welche ein \"ubergreifendes und vereinheitlichtes Arbeiten mit den Datenbest\"anden erm\"oglicht. Heutige \ac{DMS} decken eine Vielzahl von Anwendungsgebieten ab, welche an die jeweiligen Problemstellungen wie Rechtsfragen oder fachliche Anforderungen angepasst sind oder angepasst werden k\"onnen.
Ein genaues Konzept mit Anforderungen soll am Beispiel der \ac{LUBW} sowie der \ac{GAA} entstehen.
\cite{Dokumenten-Management}

Es sollen durch das Konzept auch bestehende Systeme weiter unterst\"utzt werden, wof\"ur hier eine Anbindungsm\"oglichkeit geschaffen werden soll. \"Uber eine serviceorientierte Anbindung sollen somit die verschiedensten Frontends wie Desktopanwendung, Web, Mobil usw. das neue Konzept nutzen k\"onnen. Welches \ac{ECM}-Tool f\"ur die Erstellung des Projekts am besten geeignet ist, muss analysiert werden, ergibt sich jedoch aus den zu verwaltenden Dokumenten.

F\"ur die Arbeit soll hierbei kein komplett neues \ac{DMS} mit Nutzer- und Dokumentenverwaltung entwickelt werden. Vielmehr soll evaluiert werden, welche Systeme es bereits auf dem Markt gibt, und wie sich diese f\"ur die anstehende Aufgabe eignen, vor allem im Bezug auf die hierarchische Gliederung der Metadaten, wie im Projekt vorgesehen ist. 

Zus\"atzlich zu den System der \ac{LUBW} soll das Literaturverzeichnis der \ac{ICT-ENSURE} untersucht werden. Hierf\"ur soll das Datenmodell herangezogen werden, um es wie die Systeme der \ac{LUBW} zu analysieren. 

F\"ur die \ac{ICT-ENSURE} soll jedoch keine Implementierung erfolgen.

Als sogenannter \ac{PoC} soll im Verlauf dieser Arbeit ein bestehendes Fachsystem auf der Basis einer erarbeiteten Schnittstelle nachgebaut werden. Diese Frontend soll das derzeitige \ac{FADO}-System der \ac{LUBW} nachbilden.