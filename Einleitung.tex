\section{Einleitung} \label{Einfuehrung}
Beh\"orden und Firmen haben heute immer gr\"o\ss{}ere Datenbest\"ande zu verwalten, welche schon in elektronischen Systemen vorzufinden sind. Da diese Systeme historisch bedingt gewachsen sind, entstanden im laufe der Zeit immer mehr kleinere Insell\"osungen, welche ein \"ubergreifendes Verwalten und Nutzen von Daten erschwert oder gar ganz verhindert.

Das Ziel dieser Arbeit soll es daher sein, ein Konzept sowie eine prototypische Implementierung zu ertsellen, welche ein \"ubergreifendes und vereinheitlichtes arbeiten mit den Datenbest\"anden erm\"oglicht. Dieses Konzept, soll am Beispiel der \ac{LUBW} sowie der \ac{GAA} entstehen.

Es sollen durch das Konzept auch bestehende Systeme weiter unterst\"utzt werden, wof\"ur hier eine Anbindungsm\"oglichkeit beachtet werden soll. \"Uber eine serviceorientierte Anbinung sollen somit die verschiedensten Frontends wie Desktopanwendung, Web, Mobil usw. das neue Konzept nutzen k\"onnen.

F\"ur die Arbeit soll hierbei kein komplett neues \ac{DMS} mit Nutzer und Dokumentverwaltung entwickelt werden. Vielmehr soll evaluiert werden, welche Systeme es bereits auf dem Markt gibt, und wie sich diese f\"ur die anstehnde Aufgabe eignen. 

Als sogenannter \ac{PoC} soll im Verlauf dieser Arbeit ein bestehendes Fachsystem auf der Basis der erarbeiteten Schnittstelle nachgebaut werden.