\section{Stand der Technik / existierende Konzepte} \label{Stand der Technik}
Im folgenden Kapiel werden die einzelnen Fachsysteme, der \ac{LUBW}, und deren Datenbest\"ande genauer analysiert. Welche Metdaten die einzelnen Dokumentenbest\"ande der Fachsysteme genau enthalten wird im Kapitel \ref{Analyse Datenbestaende} genauer erkl\"autert. 

Die einzelnen Fachsysteme der \ac{LUBW} sind f\"ur unterschieliche Einsatzzw\"acke entwickelt wurden, welche nun genauer anlaysiert werden.
\subsection{FADO und Untergruppen} \label{FADO}
Das \ac{FADO}-System der \ac{LUBW} erlaubt es den Nutzern nach verschiedenen Texte aus unterschielichen Themenrichtungen zu suchen und diese herunterzuladen. Hierbei stehen die Dokumente im PDF- oder HTML-Format bereit.

Die einzelnen Ver\"offentlichungen k\"onnen nach verschiedenen Kriterien durchsucht werden, wobei der Bestand an Dokumenten fortlaufend erweitert wird.
\cite{LUBW_FADO}

Im \ac{FADO}-System sind die Dokumente in die sechs Kategorien Altlasten, Boden, Natur und Landschaft, Umweltbeobachtung, Umweltforschung und Umweltinformationssysteme gegliedert. Hierbei ist zu beachten, dass ein Dokument nicht nur einem Themengebiet zugeordent werden kann, sondern durchaus unter mehreren Gebieten zu finden ist. Solche Dokumente sind im \ac{FADO} jedoch nur einmal abgespeichert und werden \"uber Relationen zu verschiedenen Kategorien zugeordent.

Die einzelnen Kategorien sind wiederum mit Untergruppen versehen, welche die Zugeh\"origkeit der Dokumente konkretisieren.

Da im Verlauf der Arbeit nicht alle Gruppen in ein neues System \"uberf\"uhrt werden k\"onnen, wird sich auf die Dokumentklassen Berichte, Urteile und Forschungsvorhaben beschr\"ankt, welche in der Kategorie Natur und Landschaft, in der Kategorie Boden beziehungsweise in der Kategorie Umweltforschung zu finden sind.

Forschungsvorhaben sind wiederum Berichte, welche sich in eine Skizze, einen Zwischen- und einen Abschlussbericht aufteilen. Hierbei geh\"oren die drei Teile zu jeweils einem Forschungsvorhaben.

\subsection{Das Document Retrieval System} \label{DRS}
Das \ac{DRS} der \ac{LUBW} ist eine Art Suchmaschine, die es erm\"oglicht verschiedene Rechtsvorschriften, Regelungen, Fachberichte und Erlasse zu suchen. Abfragen im \ac{DRS} k\"onnen auf drei Arten erfolgen. Es gibt die "`Standardsuche"', welche es erlaubt nach Inhalt oder Metadaten der Dokumente zu suchen. Die "`Titelsuche"' erlaubt es, wie der Name schon sagt nach Titeln oder gegebenenfalls nach Normtiteln der Dokumente zu schen. Als dritte M\"oglichkeit bietet das \ac{DRS} eine "`Geziehlte Suche"' an, die es erm\"oglicht Suchkriterien genau anzugeben und diese auch einzuschr\"anken.
\cite{DRS}

Zu beachten ist, dass das \ac{DRS} eine eigenst\"andiges Plattform ist, welche einen eigenen Dokumentenbestand vorh\"alt.

In Abbildung \ref{Suchmaske DRS} ist die Suchmaske der "`Geziehlten Suche"' vom \ac{DRS} zu sehen. Es werden in der Maske alle M\"oglichkeiten aufgef\"uhrt, nach denen gesucht werden kann, was alle vorhandenen Metadaten mit einschlie\ss{}t. Die meisten Metadaten, sind durch Auswahllisten beschr\"ankt und andere vom Nutzer frei mit Begrifflichkeiten gef\"ullt werden.

Zu sehen ist in den Feldern Fassung und \"Anderung, das eine Art Verwaltung von \"alteren Versionen vorgenommen wird. Diese Versionierung ist notwendig, da alte Gesetzestexte f\"ur in der Vergangenheit liegende F\"alle aufbewahrt werden m\"ussen.

\begin{figure}[!ht]
\centering
\includegraphics[width=15cm]{Bilder/Suchmaske_DRS.jpg}
\caption{Suchmaske des \ac{DRS}}
\label{Suchmaske DRS}
\centering
\end{figure}

\subsection{Naturschutz-Bildarchiv} \label{Bilddatenbank}
Im Naturschutz-Bildarchiv der \ac{LUBW} finden sich viele Bilder zu verschiedenen Themengebieten, so zum Beispiel auch zu den Gebieten Biotoptyp, Lebensraumtyp, Naturschutzgebiet und einige mehr.
\cite{Naturschutz-Bildarchiv}

Diese Themenkomplexe k\"onnen wiederum nach Stichworten durchsucht werden, wie zum Beispiel "`Aurorafalter"', was im Themengebiet "`Pflanzen- und Tierart"' Bilder von entsprechenden "`Faltern"' liefert. Auch das Bildarchiv ist ein eigenst\"andiges System, welches seinen Dokumentenbestand besitzt.


